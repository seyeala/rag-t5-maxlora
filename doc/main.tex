\documentclass[11pt]{article}
\usepackage[margin=1in]{geometry}
\usepackage{hyperref}
\usepackage{xcolor}
\usepackage{enumitem}
\usepackage{listings}
\usepackage{graphicx}
\usepackage{amsmath}

\hypersetup{
  colorlinks=true,
  linkcolor=blue,
  urlcolor=blue,
  citecolor=blue,
  pdftitle={Small-Decoder Chatbot: Middle-Block Fine-Tuning vs QLoRA (All-Free)},
  pdfauthor={Your Name}
}

% ----- Listings (code) style -----
\lstdefinestyle{py}{
  language=Python,
  basicstyle=\ttfamily\footnotesize,
  breaklines=true,
  showstringspaces=false,
  frame=single,
  numbers=none,
  keywordstyle=\bfseries,
}
\lstdefinestyle{bash}{
  language=Bash,
  basicstyle=\ttfamily\footnotesize,
  breaklines=true,
  showstringspaces=false,
  frame=single,
  numbers=none,
}
\lstset{tabsize=2}

\title{Small-Decoder \textbf{Chatbot} Homework:\\ Middle-Block Fine-Tuning vs QLoRA (All-Free Stack)}
\author{Your Name \quad|\quad Course Code}
\date{\today}

% === Added (minimal) notation for transformer components and loss ===
\makeatletter
\@ifundefined{DeclareMathOperator}{%
  \newcommand{\Enc}{\mathrm{Enc}}%
  \newcommand{\Dec}{\mathrm{Dec}}%
  \newcommand{\Emb}{\mathrm{Emb}}%
  \newcommand{\Tok}{\mathrm{Tok}}%
  \newcommand{\Trans}{\mathrm{T}}%
  \newcommand{\Loss \eqref{eq:loss}}{\mathcal{L}}%
}{%
  \DeclareMathOperator{\Enc}{Enc}
  \DeclareMathOperator{\Dec}{Dec}
  \DeclareMathOperator{\Emb}{Emb}
  \DeclareMathOperator{\Tok}{Tok}
  \DeclareMathOperator{\Trans}{T}
}
\makeatother
% === End added notation ===
\begin{document}
\maketitle

\section*{Overview}

\section*{Notation}
\begin{itemize}[leftmargin=2em]
  \item Vocabulary size $|\mathcal{V}|$, model width $d_{\text{model}}$, feed-forward width $d_{\text{ff}}$, number of heads $h$, per-head key/value width $d_k=d_v=d_{\text{model}}/h$.
  \item A source (encoder) sequence of length $n$: tokens $(s_1,\dots,s_n)$; a target (decoder) sequence of length $m$: $(t_1,\dots,t_m)$.
  \item Embedding matrix $W_E \in \mathbb{R}^{|\mathcal{V}|\times d_{\text{model}}}$, with per-token embedding $e_i = W_E[\text{token}_i] \in \mathbb{R}^{d_{\text{model}}}$.
  \item Positional Encod \eqref{eq:pe}ing $p_i \in \mathbb{R}^{d_{\text{model}}}$ (sinusoidal or learned); input to each stack is $x_i = \sqrt{d_{\text{model}}}\,e_i + p_i$.
  \item All length-$\ell$ sequences are arranged row-wise into matrices in $\mathbb{R}^{\ell\times d_{\text{model}}}$.
  \item Layer Normalization is denoted $\mathrm{LN}(\cdot)$; dropout is implied where standard.
\end{itemize}

\section{Positional Encoding}
For absolute sinusoidal encodings, position $i\!\in\!\{1,\dots,\ell\}$ and channel index $0\!\leq\!2j\!<\!d_{\text{model}}$:
\begin{align}
\mathrm{PE}(i,2j)   &= \sin\!\Big(i/10000^{\frac{2j}{d_{\text{model}}}}\Big),&
\mathrm{PE}(i,2j\!+\!1) &= \cos\!\Big(i/10000^{\frac{2j}{d_{\text{model}}}}\Big). \label{eq:pe}
\end{align}
These vectors are added to token embeddings at the input of the encoder and decoder stacks.

\section{Scaled Dot-Product Attention \eqref{eq:sdpa}}
Given queries $Q\!\in\!\mathbb{R}^{\ell_q\times d_k}$, keys $K\!\in\!\mathbb{R}^{\ell_k\times d_k}$, values $V\!\in\!\mathbb{R}^{\ell_k\times d_v}$ and an additive mask $M\!\in\!\mathbb{R}^{\ell_q\times \ell_k}$ with entries $0$ (keep) or $-\infty$ (mask), the attention map is
\begin{equation}
\mathrm{Attention}(Q,K,V;M) \;=\; \mathrm{softmax}\!\Big(\tfrac{QK^{\top}}{\sqrt{d_k}} + M\Big)\,V. \label{eq:sdpa}
\end{equation}
The \emph{padding mask} masks positions that correspond to PAD tokens; the decoder's \emph{causal (look-ahead) mask} additionally masks future positions $j>i$ so that token $t_i$ cannot attend to $t_{i+1:m}$.

\section{Multi-Head Attention \eqref{eq:mha} (MHA)}
For input $X\!\in\!\mathbb{R}^{\ell\times d_{\text{model}}}$ and head index $a\in\{1,\dots,h\}$, let
\begin{equation}
Q_a = X W_Q^{(a)},\quad K_a = X W_K^{(a)},\quad V_a = X W_V^{(a)},\qquad
W_Q^{(a)},W_K^{(a)},W_V^{(a)} \in \mathbb{R}^{d_{\text{model}}\times d_k}.
\label{eq:qkv_self}
\end{equation}
Define head outputs $H_a = \mathrm{Attention}(Q_a,K_a,V_a;M)$ with a mask $M$ as appropriate (padding or causal). Concatenate heads and project:
\begin{equation}
\mathrm{MHA}(X;M) \;=\; \mathrm{Concat}(H_1,\dots,H_h)\,W_O,\qquad W_O\in\mathbb{R}^{hd_v\times d_{\text{model}}}. \label{eq:mha}
\end{equation}

\section{Position-wise Feed-Forward Network \eqref{eq:ffn} (FFN)}
Applied independently at each position $i$:
\begin{equation}
\mathrm{FFN}(x_i) \;=\; W_2\,\sigma(W_1 x_i + b_1) + b_2,\qquad
W_1\in\mathbb{R}^{d_{\text{model}}\times d_{\text{ff}}},\;W_2\in\mathbb{R}^{d_{\text{ff}}\times d_{\text{model}}},
\label{eq:ffn}
\end{equation}
where $\sigma$ is typically ReLU in the original Transformer (GELU is a common modern variant).

\section{Residual + Layer Normalization Blocks}
The original (``Post-LN \eqref{eq:postln}'') Transformer applies residual connections followed by Layer Normalization after each sub-layer:
\begin{align}
\mathrm{PostLNBlock}(X,\mathcal{F}) &= \mathrm{LN}\!\big(X + \mathrm{Dropout}(\mathcal{F}(X))\big).\label{eq:postln}
\end{align}
Many modern implementations use the ``Pre-LN'' variant, placing Layer Normalization \emph{before} each sub-layer:
\begin{align}
\mathrm{PreLNBlock}(X,\mathcal{F}) &= X + \mathrm{Dropout}\!\big(\mathcal{F}(\mathrm{LN}(X))\big).\label{eq:preln}
\end{align}
Here $\mathcal{F}$ is either MHA or FFN.

\section{Encoder}
The encoder is a stack of $L$ identical layers. Let $S\in\mathbb{R}^{n\times d_{\text{model}}}$ be the embedded source inputs (embeddings plus positions). For layer $\ell=1,\dots,L$:
\begin{align}
U^{(\ell)} &= \mathrm{PostLNBlock}\!\left(S^{(\ell-1)},\, X\mapsto \mathrm{MHA}(X;M_{\text{pad}})\right), \\
S^{(\ell)} &= \mathrm{PostLNBlock}\!\left(U^{(\ell)},\; \mathrm{FFN}\right),
\label{eq:postln_stack}
\end{align}
with $S^{(0)}=S$. (Replace with \eqref{eq:preln} to obtain the Pre-LN encoder.) The final encoder memory is $E \coloneqq S^{(L)}$.

\section{Decoder}
The decoder is also a stack of $L$ layers. Let $T\in\mathbb{R}^{m\times d_{\text{model}}}$ be the embedded target inputs (shifted right during training). For layer $\ell=1,\dots,L$:
\begin{align}
U^{(\ell)} &= \mathrm{PostLNBlock}\!\left(T^{(\ell-1)},\, X\mapsto \mathrm{MHA}(X;M_{\text{causal}} + M_{\text{pad}})\right), \\
V^{(\ell)} &= \mathrm{PostLNBlock}\!\left(U^{(\ell)},\, X\mapsto \mathrm{MHA}\big(X;M_{\text{enc\text{-}dec}}\big)\right), \\
T^{(\ell)} &= \mathrm{PostLNBlock}\!\left(V^{(\ell)},\; \mathrm{FFN}\right),
\label{eq:postln_stack_2}
\end{align}
with $T^{(0)}=T$. The cross-attention mask $M_{\text{enc\text{-}dec}}$ applies padding from the source; queries come from the decoder stream while keys/values come from the encoder memory $E$:
\begin{equation}
\text{Cross-Attention:}\quad Q_a = V^{(\ell)}W_Q^{(a)},\qquad K_a = E W_K^{(a)},\qquad V_a = E W_V^{(a)}.
\label{eq:qkv_cross}
\end{equation}

\section{Output Projection, Softmax, and Loss}
Let $H = T^{(L)}\in\mathbb{R}^{m\times d_{\text{model}}}$ be final decoder states. Following the original paper, it is standard to \emph{tie} the output projection to the input embeddings: logits are
\begin{equation}
Z = H W_E^{\top} + b_{\text{lm}},\qquad Z\in\mathbb{R}^{m\times |\mathcal{V}|}.
\label{eq:lm_logits}
\end{equation}
Per-token probabilities are $p_t = \mathrm{softmax}(Z_t)$. With label smoothing $\varepsilon\in[0,1)$, the smoothed target distribution for a true label $y_t$ is
\begin{equation}
\tilde{y}_t = (1-\varepsilon)\,\mathrm{onehot}(y_t) + \frac{\varepsilon}{|\mathcal{V}|}\,\bm{1}.
\label{eq:label_smoothing}
\end{equation}
The masked Cross-Entropy Loss over non-PAD positions $\mathcal{I}$ is
\begin{equation}
\mathcal{L} = - \frac{1}{|\mathcal{I}|}\sum_{t\in\mathcal{I}} \tilde{y}_t^{\top}\log p_t. \label{eq:loss}
\end{equation}

\section{Complexities and Shapes}
Self-attention in a length-$\ell$ sequence costs $\mathcal{O}(\ell^2 d_{\text{model}})$ time and $\mathcal{O}(\ell^2)$ memory for the attention map. Per-head shapes are $Q,K\in\mathbb{R}^{\ell\times d_k}$, $V\in\mathbb{R}^{\ell\times d_v}$; concatenation gives $\ell\times (hd_v)$ before the output projection.

\section{Training Regime (Canonical)}
The canonical setup: ReLU FFN, dropout in each sub-layer, Adam with warmup, label smoothing $\varepsilon=0.1$, and Post-LN residual blocks. Modern variants often prefer Pre-LN for stability and use GELU in the FFN.
Layer Normalization is as defined by Ba \emph{et~al.}.

\paragraph{Summary of terminology (checked against the literature).}
\emph{Scaled Dot-Product Attention}, \emph{Multi-Head Attention}, \emph{Position-wise Feed-Forward Network}, \emph{Add \& Norm (residual + Layer Normalization)}, \emph{causal/look-ahead mask}, \emph{padding mask}, \emph{encoder--decoder (cross) attention}, \emph{Post-LN vs.\ Pre-LN}, \emph{label smoothing}, and \emph{weight tying} are the standard terms used in the primary sources.

\vspace{1em}
\noindent\textbf{End-to-end forward pass (decoder at time $t$).} With teacher forcing during training:
\begin{align}
S &= \big(\sqrt{d_{\text{model}}}E[s_1{:}s_n]\big) + P[1{:}n], \\
E &= \mathrm{Encoder}(S), \\
T_{1{:}t} &= \big(\sqrt{d_{\text{model}}}E[t_0{:}t_{t-1}]\big) + P[0{:}t-1], \qquad (\text{BOS}=t_0) \\
H_{1{:}t} &= \mathrm{Decoder}(T_{1{:}t}, E), \\
p_t &= \mathrm{softmax}\!\left(H_t W_E^{\top} + b_{\text{lm}}\right).
\label{eq:seq2seq_flow}
\end{align}

You will build and evaluate a \textbf{chatbot} using a \textbf{small decoder LLM} and compare three training strategies under free-tier constraints:
\begin{enumerate}[label=\textbf{A.\,}]
  \item \textbf{Middle-block fine-tuning (structural)}: freeze embeddings and top/bottom layers; train only the \emph{middle} transformer blocks $+$ \texttt{lm\_head}.
  \item \textbf{QLoRA (PEFT)}: 4-bit quantized loading with \textbf{LoRA} adapters applied to attention/MLP projections on the same middle slice \cite{dettmers2023qlora,hu2021lora}.
  \item \textbf{Tiny baseline}: either head-only training or a minimal LoRA on just the \emph{top} few layers.
\end{enumerate}

You will deploy a \textbf{Gradio} chatbot UI, push weights to the \textbf{Hugging Face Hub}, and submit a short report. All tools and resources listed are free-tier compatible \cite{wolf2020transformers,gradio,bitsandbytes,hfhub}.

\section*{Models \& Tasks (Choose One Model and One Task)}
\paragraph{Model (Apache-2.0):}
\begin{itemize}
  \item \texttt{TinyLlama/TinyLlama-1.1B-Chat-v1.0} \cite{tinyllama} \quad or \quad \texttt{Qwen/Qwen2.5-0.5B-Instruct} \cite{qwen25}.
\end{itemize}

\paragraph{Task:}
\begin{enumerate}[label=\arabic*)]
  \item \textbf{Short-form instruction tuning} (single-turn): e.g., \texttt{yahma/alpaca-cleaned} \cite{alpaca-cleaned} or \texttt{databricks/databricks-dolly-15k} \cite{dolly15k}.\\
        \emph{Note: Dolly-15k is CC BY-SA; attribute and share alike.}
  \item \textbf{Sentiment-as-generation}: e.g., GLUE \texttt{sst2} (frame as ``\emph{Label?}'' $\rightarrow$ \emph{positive/negative}) \cite{glue}.
\end{enumerate}

\section*{Free Tooling}
\begin{itemize}
  \item \textbf{Compute}: Google Colab (free) or Kaggle (free) GPU.
  \item \textbf{Python libs}: \texttt{transformers}, \texttt{datasets}, \texttt{peft}, \texttt{accelerate}, \texttt{bitsandbytes}, \texttt{evaluate}, \texttt{gradio}, \texttt{huggingface\_hub} \cite{wolf2020transformers,hfhub,bitsandbytes,gradio}.
  \item \textbf{Weights \& demo hosting}: Hugging Face Hub and optional Space (free CPU).
\end{itemize}

\section*{What You Will Compare}
Train \textbf{three variants} under the \textbf{same budget} (batch/epochs/seq length):
\begin{enumerate}[label=\textbf{V\arabic*)}]
  \item \textbf{Middle-block FT}: Train only the middle third of layers (plus \texttt{lm\_head}).
  \item \textbf{QLoRA on middle layers}: 4-bit loading with LoRA on attention/\texttt{gate/up/down/o\_proj}.
  \item \textbf{Tiny baseline}: head-only or LoRA on the top 1--2 layers.
\end{enumerate}
Report: trainable parameters, VRAM (peak), wall time, and quality metrics.

\section*{Data Preparation (Small \& Free)}
\begin{itemize}
  \item Subsample $\sim$5{,}000 examples for instruction tuning or the training split of \texttt{sst2} for classification-as-generation.
  \item Format single-turn prompts consistently (see code).
  \item Keep \texttt{max\_length} $\leq$ 512 to fit free GPUs.
\end{itemize}

\section*{Evaluation}
\begin{itemize}
  \item \textbf{Automatic}: For instruction tuning, simple \emph{Exact Match} / token-level F1 on a small held-out set. For \texttt{sst2}, exact label match (``positive''/``negative''). Token-F1 uses $P=|y\cap\hat y|/|\hat y|$, $R=|y\cap\hat y|/|y|$, $F_1=2PR/(P{+}R)$.
  \item \textbf{Human rubric} (30 prompts): \emph{Helpfulness} 0--2, \emph{Faithfulness} 0--2 (average).
  \item \textbf{Efficiency}: trainable parameter count, peak VRAM, wall time / epoch. For LoRA on a weight $W$ of shape $d\times k$ and rank $r$, trainable params per module $=r(d{+}k)$ \cite{hu2021lora}.
\end{itemize}

\section*{Step-by-Step (Colab/Kaggle)}
\paragraph{Install \& login (free):}
\begin{lstlisting}[style=bash]
pip -q install transformers datasets peft accelerate bitsandbytes evaluate gradio huggingface_hub
huggingface-cli login
\end{lstlisting}

\paragraph{Load dataset and format single-turn chat:}
\begin{lstlisting}[style=py]
from datasets import load_dataset

ds = load_dataset("yahma/alpaca-cleaned")  # or "glue", "sst2"
def to_chat(example):
    return {
        "prompt": f"### Instruction:\n{example['instruction']}\n"
                  f"### Input:\n{example.get('input','')}\n### Response:\n",
        "response": example["output"]
    }
train = ds["train"].select(range(5000)).map(to_chat)  # small for free GPU
\end{lstlisting}

\paragraph{Load model in 4-bit (QLoRA-ready):}
\begin{lstlisting}[style=py]
import torch
from transformers import AutoModelForCausalLM, AutoTokenizer, BitsAndBytesConfig

MODEL = "TinyLlama/TinyLlama-1.1B-Chat-v1.0"  # or "Qwen/Qwen2.5-0.5B-Instruct"
bnb = BitsAndBytesConfig(load_in_4bit=True, bnb_4bit_compute_dtype=torch.bfloat16)

tok = AutoTokenizer.from_pretrained(MODEL, use_fast=True)
tok.pad_token = tok.eos_token
model = AutoModelForCausalLM.from_pretrained(MODEL, quantization_config=bnb, device_map="auto")
\end{lstlisting}

\paragraph{V1: Middle-block fine-tuning (freeze others):}
Objective (freeze $\theta_{\ell\notin[s,e]}$): $\displaystyle \min_{\theta_{s:e}}\, -\sum_{(x,y)\in\mathcal{D}}\sum_t\log p_\theta(y_t\mid y_{<t},x)$ \cite{brown2020gpt3,bishop2006prml}.
\begin{lstlisting}[style=py]
# Freeze everything first
for p in model.parameters():
    p.requires_grad = False

# Choose middle slice (e.g., middle third)
L = len(model.model.layers)
mid_idx = range(L//3, 2*L//3)
for i in mid_idx:
    for p in model.model.layers[i].parameters():
        p.requires_grad = True

# Always train the lm_head
for p in model.lm_head.parameters():
    p.requires_grad = True
\end{lstlisting}

% === Inserted: clarify train/freeze for V1 (as implemented) ===
\noindent\textbf{Train vs.\ freeze (as implemented):}
\begin{itemize}
  \item \textbf{Train:} \texttt{lm\_head} only.
  \item \textbf{Freeze:} embeddings (token and position, if any) and all transformer blocks.
\end{itemize}
\emph{Reason:} the model is loaded in 4-bit (\texttt{BitsAndBytes}); in this configuration only the non-quantized \texttt{lm\_head} parameters actually receive updates; unfreezing middle blocks has no effect.
% === End insert V1 ===



\paragraph{V2: QLoRA on the same middle slice:}
Math: apply LoRA $\Delta W=\frac{\alpha}{r}BA$ (rank $r$) to 4-bit base $W_q=Q_4(W_0)$; update only $A,B$, forward $h=(W_q+\Delta W)x$ \cite{hu2021lora,dettmers2023qlora}.
\begin{lstlisting}[style=py]
from peft import LoraConfig, get_peft_model

target_modules = ["q_proj","k_proj","v_proj","o_proj","gate_proj","up_proj","down_proj"]
peft_cfg = LoraConfig(r=8, lora_alpha=16, lora_dropout=0.05, target_modules=target_modules)
model = get_peft_model(model, peft_cfg)

# (Optional) If you want to limit LoRA to certain layers,
# configure target module names accordingly or filter by name when creating adapters.
model.print_trainable_parameters()
\end{lstlisting}

% === Inserted: clarify train/freeze for V2 (as implemented) ===
\noindent\textbf{Train vs.\ freeze (as implemented):}
\begin{itemize}
  \item \textbf{Train:} LoRA $A/B$ on \texttt{q\_proj}, \texttt{k\_proj}, \texttt{v\_proj}, \texttt{o\_proj}, \texttt{gate\_proj}, \texttt{up\_proj}, \texttt{down\_proj} in \emph{all} layers; (optionally) \texttt{lm\_head}.
  \item \textbf{Freeze:} 4-bit base weights everywhere; no LoRA restriction to the middle slice.
\end{itemize}
% === End insert V2 ===



\paragraph{V3: Tiny baseline (head-only or top-2 layers LoRA):}
Head-only: optimize only $W_{\text{out}}$ with the same loss; $p=\mathrm{softmax}(W_{\text{out}}h)$, others frozen \cite{brown2020gpt3}.
\begin{lstlisting}[style=py]
# Head-only example (simple baseline):
for p in model.parameters(): p.requires_grad = False
for p in model.lm_head.parameters(): p.requires_grad = True
\end{lstlisting}

% === Inserted: clarify train/freeze for V3 (as implemented) ===
\noindent\textbf{Train vs.\ freeze (as implemented):}
\begin{itemize}
  \item \textbf{Train:} \texttt{lm\_head} only (head-only baseline). \emph{(Optional alternative)}: LoRA $A/B$ only on the top-2 layers' projections.
  \item \textbf{Freeze:} everything else (base weights and non-top-2 layers/adapters).
\end{itemize}
% === End insert V3 ===



\paragraph{Tokenize and train (keep it tiny for free GPUs):}
\begin{lstlisting}[style=py]
from transformers import DataCollatorForLanguageModeling, Trainer, TrainingArguments

def pack(example):
    text = example["prompt"] + example["response"] + tok.eos_token
    out = tok(text, truncation=True, max_length=512)
    out["labels"] = out["input_ids"].copy()
    return out

tok_train = train.map(pack, remove_columns=train.column_names)
collator = DataCollatorForLanguageModeling(tok, mlm=False)

args = TrainingArguments(
    output_dir="./out",
    num_train_epochs=2,
    per_device_train_batch_size=1,
    gradient_accumulation_steps=16,
    learning_rate=1e-4,
    bf16=True,
    logging_steps=25,
    save_steps=500,
    save_total_limit=2,
)

trainer = Trainer(model=model, args=args, train_dataset=tok_train, data_collator=collator)
trainer.train()
\end{lstlisting}

\paragraph{Push weights to Hugging Face Hub:}
\begin{lstlisting}[style=py]
repo = "your-hf-handle/tinyllama-middleft-alpaca"  # change per variant
trainer.model.push_to_hub(repo)
tok.push_to_hub(repo)
\end{lstlisting}

\paragraph{Gradio chatbot (switch between variants):}
\begin{lstlisting}[style=py]
import gradio as gr
import torch
from transformers import AutoTokenizer, AutoModelForCausalLM

choices = {
    "Middle-FT": "your-hf-handle/tinyllama-middleft-alpaca",
    "QLoRA":     "your-hf-handle/tinyllama-qlora-alpaca",
    "Baseline":  "your-hf-handle/tinyllama-headonly-alpaca"
}

state = {"name": list(choices.values())[0]}
tok = AutoTokenizer.from_pretrained(state["name"], use_fast=True)
lm  = AutoModelForCausalLM.from_pretrained(state["name"], torch_dtype=torch.bfloat16, device_map="auto")

def switch(model_key):
    name = choices[model_key]
    global tok, lm
    tok = AutoTokenizer.from_pretrained(name, use_fast=True)
    lm  = AutoModelForCausalLM.from_pretrained(name, torch_dtype=torch.bfloat16, device_map="auto")
    return f"Loaded {model_key}"

def chat_fn(history, msg):
    prompt = f"### Instruction:\\n{msg}\\n### Response:\\n"
    inp = tok(prompt, return_tensors="pt").to(lm.device)
    out = lm.generate(**inp, max_new_tokens=256, do_sample=False)
    ans = tok.decode(out[0][inp['input_ids'].shape[1]:], skip_special_tokens=True)
    history = (history or []) + [(msg, ans)]
    return history, ""

with gr.Blocks() as demo:
    gr.Markdown("# Small-Decoder Chatbot — Middle-Block FT vs QLoRA")
    dd = gr.Dropdown(list(choices.keys()), value="Middle-FT", label="Model")
    status = gr.Markdown()
    gr.Button("Load").click(switch, dd, status)
    chat = gr.Chatbot()
    box  = gr.Textbox(placeholder="Ask a short instruction...")
    box.submit(chat_fn, [chat, box], [chat, box])

demo.launch()
\end{lstlisting}

\section*{Deliverables}
\begin{enumerate}[label=\arabic*.]
  \item \textbf{Three HF repos} (one per variant) with README (task, data, hyperparams, trainable params).
  \item \textbf{Gradio demo}: link to a Hugging Face Space or screenshots.
  \item \textbf{1-page report}: metrics table (quality vs.\ trainable params vs.\ time), rubric scores, and 3 failure cases.
\end{enumerate}

\section*{Grading Rubric (Suggested)}
\begin{center}
\begin{tabular}{l r}
Working chatbot \& deployment & 20\%\\
Middle-block FT \& one baseline (QLoRA or head-only) & 35\%\\
Fair comparison (same budget) \& metrics & 25\%\\
HF Hub repos \& documentation & 10\%\\
Report clarity \& insights & 10\%\\
\hline
\textbf{Total} & \textbf{100\%}
\end{tabular}
\end{center}

\section*{Tips \& Pitfalls}
\begin{itemize}
  \item Keep sequence length $\leq$ 512, use gradient accumulation, and \texttt{bf16} to fit free GPUs.
  \item Middle-block FT can overfit small data: fewer steps or early stopping.
  \item For QLoRA, ensure 4-bit loading with \texttt{bitsandbytes}; include attention and MLP projections in \texttt{target\_modules}.
  \item Respect dataset licenses (e.g., CC BY-SA for Dolly-15k).
\end{itemize}

\section*{License Note}
All recommended models are Apache-2.0. Datasets have their own licenses; include attribution and follow share-alike where required.





% ========== References (moved to end) ==========
\bibliographystyle{plain}
\begin{thebibliography}{99}
\bibitem{vaswani2017}
Ashish Vaswani, Noam Shazeer, Niki Parmar, Jakob Uszkoreit, Llion Jones, Aidan N.~Gomez, Lukasz Kaiser, and Illia Polosukhin.
\newblock Attention Is All You Need.
\newblock \emph{NeurIPS}, 2017.

\bibitem{ba2016layernorm}
Jimmy~Lei Ba, Jamie~Ryan Kiros, and Geoffrey~E. Hinton.
\newblock Layer Normalization.
\newblock \emph{arXiv:1607.06450}, 2016.

\bibitem{xiong2020layernorm}
Ruibin Xiong, Yunchang Yang, Di He, Kai Zheng, Shuxin Zheng, Chen Xing, Huishuai Zhang, Yanyan Lan, Liwei Wang, and Tie-Yan Liu.
\newblock On Layer Normalization in the Transformer Architecture.
\newblock In \emph{Proceedings of ICML}, 2020.
\cite{vaswani2017attention}; we fine-tune via LoRA adapters \cite{hu2021lora}, optimizing token-level cross-entropy \cite{bishop2006prml}.

\bibitem{hu2021lora}
Hu, Edward J., et al. \emph{LoRA: Low-Rank Adaptation of Large Language Models}. 2021. \href{https://arxiv.org/abs/2106.09685}{arXiv:2106.09685}.

\bibitem{dettmers2023qlora}
Dettmers, Tim, et al. \emph{QLoRA: Efficient Finetuning of Quantized LLMs}. 2023. \href{https://arxiv.org/abs/2305.14314}{arXiv:2305.14314}.

\bibitem{wolf2020transformers}
Wolf, Thomas, et al. \emph{Transformers: State-of-the-Art Natural Language Processing}. EMNLP 2020 Demos. \href{https://arxiv.org/abs/1910.03771}{arXiv:1910.03771}.

\bibitem{hfhub}
Hugging Face Hub Documentation. \href{https://huggingface.co/docs/hub/index}{https://huggingface.co/docs/hub/index}.

\bibitem{bitsandbytes}
Dettmers, Tim. \emph{bitsandbytes: 8-bit optimizers and quantization}. \href{https://github.com/TimDettmers/bitsandbytes}{https://github.com/TimDettmers/bitsandbytes}.

\bibitem{gradio}
Abid, A., Abdalla, A., Abid, A., Khan, D., & Zou, J. \emph{Gradio: Hassle-Free Machine Learning Web Apps}. 2019. \href{https://arxiv.org/abs/1905.04227}{arXiv:1905.04227}. See also: \href{https://www.gradio.app}{https://www.gradio.app}.

\bibitem{tinyllama}
\emph{TinyLlama/TinyLlama-1.1B-Chat-v1.0} (model card). Hugging Face. \href{https://huggingface.co/TinyLlama/TinyLlama-1.1B-Chat-v1.0}{https://huggingface.co/TinyLlama/TinyLlama-1.1B-Chat-v1.0}.

\bibitem{qwen25}
\emph{Qwen/Qwen2.5-0.5B-Instruct} (model card). Hugging Face. \href{https://huggingface.co/Qwen/Qwen2.5-0.5B-Instruct}{https://huggingface.co/Qwen/Qwen2.5-0.5B-Instruct}.

\bibitem{alpaca-cleaned}
\emph{yahma/alpaca-cleaned} (dataset). Hugging Face. \href{https://huggingface.co/datasets/yahma/alpaca-cleaned}{https://huggingface.co/datasets/yahma/alpaca-cleaned}.

\bibitem{dolly15k}
\emph{databricks/databricks-dolly-15k} (dataset). Hugging Face. \href{https://huggingface.co/datasets/databricks/databricks-dolly-15k}{https://huggingface.co/datasets/databricks/databricks-dolly-15k}.

\bibitem{glue}
Wang, Alex, et al. \emph{GLUE: A Multi-Task Benchmark and Analysis Platform for Natural Language Understanding}. 2018. \href{https://gluebenchmark.com/}{https://gluebenchmark.com/}. Hugging Face dataset: \href{https://huggingface.co/datasets/glue}{https://huggingface.co/datasets/glue}.

\bibitem{brown2020gpt3}
Brown, Tom B., et al. \emph{Language Models are Few-Shot Learners}. Advances in Neural Information Processing Systems 33 (NeurIPS), 2020. \href{https://arxiv.org/abs/2005.14165}{arXiv:2005.14165}.

\bibitem{vaswani2017attention}
Vaswani, Ashish, et al. \emph{Attention Is All You Need}. NeurIPS 2017. \href{https://arxiv.org/abs/1706.03762}{arXiv:1706.03762}.

\bibitem{bishop2006prml}
Bishop, Christopher M. \emph{Pattern Recognition and Machine Learning}. Springer, 2006. \href{https://doi.org/10.1007/978-0-387-31073-2}{doi:10.1007/978-0-387-31073-2}.

\end{thebibliography}


\end{document}


% === Inserted: one-screen summary of train/freeze by variant (as implemented) ===
\begin{table}[t]
\centering
\begin{tabular}{l l p{0.40\linewidth} p{0.40\linewidth}}
\textbf{Variant} & \textbf{Load} & \textbf{Trainable} & \textbf{Frozen} \\\hline
V1 (``middle-block'') & 4-bit base & \texttt{lm\_head} only (unfreezing middle blocks has no effect under 4-bit) & Embeddings; all transformer blocks (4-bit) \\
V2 (QLoRA) & 4-bit + LoRA & LoRA $A/B$ on $q/k/v/o$ and gate/up/down in \emph{all} layers; (optional) \texttt{lm\_head} & Base weights everywhere; no middle-slice restriction \\
V3 (head-only) & 4-bit & \texttt{lm\_head} only & Everything else \\
V3 (top-2 LoRA, optional) & 4-bit + LoRA & LoRA $A/B$ on target projections in last 2 layers; (optional) \texttt{lm\_head} & Base and adapters outside top-2 \\
\end{tabular}
\end{table}
% === End summary table ===

